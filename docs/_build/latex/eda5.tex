% Generated by Sphinx.
\def\sphinxdocclass{report}
\documentclass[letterpaper,10pt,english]{sphinxmanual}
\usepackage[utf8]{inputenc}
\DeclareUnicodeCharacter{00A0}{\nobreakspace}
\usepackage{cmap}
\usepackage[T1]{fontenc}
\usepackage{babel}
\usepackage{times}
\usepackage[Bjarne]{fncychap}
\usepackage{longtable}
\usepackage{sphinx}
\usepackage{multirow}

\addto\captionsenglish{\renewcommand{\figurename}{Fig. }}
\addto\captionsenglish{\renewcommand{\tablename}{Table }}
\floatname{literal-block}{Listing }



\title{EDA5 Project Documentation}
\date{November 10, 2015}
\release{0.1}
\author{Vasja Pavlovič}
\newcommand{\sphinxlogo}{}
\renewcommand{\releasename}{Release}
\makeindex

\makeatletter
\def\PYG@reset{\let\PYG@it=\relax \let\PYG@bf=\relax%
    \let\PYG@ul=\relax \let\PYG@tc=\relax%
    \let\PYG@bc=\relax \let\PYG@ff=\relax}
\def\PYG@tok#1{\csname PYG@tok@#1\endcsname}
\def\PYG@toks#1+{\ifx\relax#1\empty\else%
    \PYG@tok{#1}\expandafter\PYG@toks\fi}
\def\PYG@do#1{\PYG@bc{\PYG@tc{\PYG@ul{%
    \PYG@it{\PYG@bf{\PYG@ff{#1}}}}}}}
\def\PYG#1#2{\PYG@reset\PYG@toks#1+\relax+\PYG@do{#2}}

\expandafter\def\csname PYG@tok@cs\endcsname{\def\PYG@tc##1{\textcolor[rgb]{0.25,0.50,0.56}{##1}}\def\PYG@bc##1{\setlength{\fboxsep}{0pt}\colorbox[rgb]{1.00,0.94,0.94}{\strut ##1}}}
\expandafter\def\csname PYG@tok@c\endcsname{\let\PYG@it=\textit\def\PYG@tc##1{\textcolor[rgb]{0.25,0.50,0.56}{##1}}}
\expandafter\def\csname PYG@tok@na\endcsname{\def\PYG@tc##1{\textcolor[rgb]{0.25,0.44,0.63}{##1}}}
\expandafter\def\csname PYG@tok@ne\endcsname{\def\PYG@tc##1{\textcolor[rgb]{0.00,0.44,0.13}{##1}}}
\expandafter\def\csname PYG@tok@nt\endcsname{\let\PYG@bf=\textbf\def\PYG@tc##1{\textcolor[rgb]{0.02,0.16,0.45}{##1}}}
\expandafter\def\csname PYG@tok@c1\endcsname{\let\PYG@it=\textit\def\PYG@tc##1{\textcolor[rgb]{0.25,0.50,0.56}{##1}}}
\expandafter\def\csname PYG@tok@gs\endcsname{\let\PYG@bf=\textbf}
\expandafter\def\csname PYG@tok@sh\endcsname{\def\PYG@tc##1{\textcolor[rgb]{0.25,0.44,0.63}{##1}}}
\expandafter\def\csname PYG@tok@no\endcsname{\def\PYG@tc##1{\textcolor[rgb]{0.38,0.68,0.84}{##1}}}
\expandafter\def\csname PYG@tok@mo\endcsname{\def\PYG@tc##1{\textcolor[rgb]{0.13,0.50,0.31}{##1}}}
\expandafter\def\csname PYG@tok@il\endcsname{\def\PYG@tc##1{\textcolor[rgb]{0.13,0.50,0.31}{##1}}}
\expandafter\def\csname PYG@tok@kn\endcsname{\let\PYG@bf=\textbf\def\PYG@tc##1{\textcolor[rgb]{0.00,0.44,0.13}{##1}}}
\expandafter\def\csname PYG@tok@mi\endcsname{\def\PYG@tc##1{\textcolor[rgb]{0.13,0.50,0.31}{##1}}}
\expandafter\def\csname PYG@tok@vg\endcsname{\def\PYG@tc##1{\textcolor[rgb]{0.73,0.38,0.84}{##1}}}
\expandafter\def\csname PYG@tok@err\endcsname{\def\PYG@bc##1{\setlength{\fboxsep}{0pt}\fcolorbox[rgb]{1.00,0.00,0.00}{1,1,1}{\strut ##1}}}
\expandafter\def\csname PYG@tok@kt\endcsname{\def\PYG@tc##1{\textcolor[rgb]{0.56,0.13,0.00}{##1}}}
\expandafter\def\csname PYG@tok@cm\endcsname{\let\PYG@it=\textit\def\PYG@tc##1{\textcolor[rgb]{0.25,0.50,0.56}{##1}}}
\expandafter\def\csname PYG@tok@nc\endcsname{\let\PYG@bf=\textbf\def\PYG@tc##1{\textcolor[rgb]{0.05,0.52,0.71}{##1}}}
\expandafter\def\csname PYG@tok@ge\endcsname{\let\PYG@it=\textit}
\expandafter\def\csname PYG@tok@gh\endcsname{\let\PYG@bf=\textbf\def\PYG@tc##1{\textcolor[rgb]{0.00,0.00,0.50}{##1}}}
\expandafter\def\csname PYG@tok@sx\endcsname{\def\PYG@tc##1{\textcolor[rgb]{0.78,0.36,0.04}{##1}}}
\expandafter\def\csname PYG@tok@gt\endcsname{\def\PYG@tc##1{\textcolor[rgb]{0.00,0.27,0.87}{##1}}}
\expandafter\def\csname PYG@tok@nn\endcsname{\let\PYG@bf=\textbf\def\PYG@tc##1{\textcolor[rgb]{0.05,0.52,0.71}{##1}}}
\expandafter\def\csname PYG@tok@sd\endcsname{\let\PYG@it=\textit\def\PYG@tc##1{\textcolor[rgb]{0.25,0.44,0.63}{##1}}}
\expandafter\def\csname PYG@tok@mf\endcsname{\def\PYG@tc##1{\textcolor[rgb]{0.13,0.50,0.31}{##1}}}
\expandafter\def\csname PYG@tok@sr\endcsname{\def\PYG@tc##1{\textcolor[rgb]{0.14,0.33,0.53}{##1}}}
\expandafter\def\csname PYG@tok@go\endcsname{\def\PYG@tc##1{\textcolor[rgb]{0.20,0.20,0.20}{##1}}}
\expandafter\def\csname PYG@tok@nf\endcsname{\def\PYG@tc##1{\textcolor[rgb]{0.02,0.16,0.49}{##1}}}
\expandafter\def\csname PYG@tok@s2\endcsname{\def\PYG@tc##1{\textcolor[rgb]{0.25,0.44,0.63}{##1}}}
\expandafter\def\csname PYG@tok@m\endcsname{\def\PYG@tc##1{\textcolor[rgb]{0.13,0.50,0.31}{##1}}}
\expandafter\def\csname PYG@tok@nl\endcsname{\let\PYG@bf=\textbf\def\PYG@tc##1{\textcolor[rgb]{0.00,0.13,0.44}{##1}}}
\expandafter\def\csname PYG@tok@gu\endcsname{\let\PYG@bf=\textbf\def\PYG@tc##1{\textcolor[rgb]{0.50,0.00,0.50}{##1}}}
\expandafter\def\csname PYG@tok@gd\endcsname{\def\PYG@tc##1{\textcolor[rgb]{0.63,0.00,0.00}{##1}}}
\expandafter\def\csname PYG@tok@vi\endcsname{\def\PYG@tc##1{\textcolor[rgb]{0.73,0.38,0.84}{##1}}}
\expandafter\def\csname PYG@tok@sc\endcsname{\def\PYG@tc##1{\textcolor[rgb]{0.25,0.44,0.63}{##1}}}
\expandafter\def\csname PYG@tok@mh\endcsname{\def\PYG@tc##1{\textcolor[rgb]{0.13,0.50,0.31}{##1}}}
\expandafter\def\csname PYG@tok@si\endcsname{\let\PYG@it=\textit\def\PYG@tc##1{\textcolor[rgb]{0.44,0.63,0.82}{##1}}}
\expandafter\def\csname PYG@tok@cp\endcsname{\def\PYG@tc##1{\textcolor[rgb]{0.00,0.44,0.13}{##1}}}
\expandafter\def\csname PYG@tok@kr\endcsname{\let\PYG@bf=\textbf\def\PYG@tc##1{\textcolor[rgb]{0.00,0.44,0.13}{##1}}}
\expandafter\def\csname PYG@tok@gi\endcsname{\def\PYG@tc##1{\textcolor[rgb]{0.00,0.63,0.00}{##1}}}
\expandafter\def\csname PYG@tok@ss\endcsname{\def\PYG@tc##1{\textcolor[rgb]{0.32,0.47,0.09}{##1}}}
\expandafter\def\csname PYG@tok@kp\endcsname{\def\PYG@tc##1{\textcolor[rgb]{0.00,0.44,0.13}{##1}}}
\expandafter\def\csname PYG@tok@kc\endcsname{\let\PYG@bf=\textbf\def\PYG@tc##1{\textcolor[rgb]{0.00,0.44,0.13}{##1}}}
\expandafter\def\csname PYG@tok@nd\endcsname{\let\PYG@bf=\textbf\def\PYG@tc##1{\textcolor[rgb]{0.33,0.33,0.33}{##1}}}
\expandafter\def\csname PYG@tok@nv\endcsname{\def\PYG@tc##1{\textcolor[rgb]{0.73,0.38,0.84}{##1}}}
\expandafter\def\csname PYG@tok@vc\endcsname{\def\PYG@tc##1{\textcolor[rgb]{0.73,0.38,0.84}{##1}}}
\expandafter\def\csname PYG@tok@k\endcsname{\let\PYG@bf=\textbf\def\PYG@tc##1{\textcolor[rgb]{0.00,0.44,0.13}{##1}}}
\expandafter\def\csname PYG@tok@se\endcsname{\let\PYG@bf=\textbf\def\PYG@tc##1{\textcolor[rgb]{0.25,0.44,0.63}{##1}}}
\expandafter\def\csname PYG@tok@ni\endcsname{\let\PYG@bf=\textbf\def\PYG@tc##1{\textcolor[rgb]{0.84,0.33,0.22}{##1}}}
\expandafter\def\csname PYG@tok@bp\endcsname{\def\PYG@tc##1{\textcolor[rgb]{0.00,0.44,0.13}{##1}}}
\expandafter\def\csname PYG@tok@w\endcsname{\def\PYG@tc##1{\textcolor[rgb]{0.73,0.73,0.73}{##1}}}
\expandafter\def\csname PYG@tok@gr\endcsname{\def\PYG@tc##1{\textcolor[rgb]{1.00,0.00,0.00}{##1}}}
\expandafter\def\csname PYG@tok@s\endcsname{\def\PYG@tc##1{\textcolor[rgb]{0.25,0.44,0.63}{##1}}}
\expandafter\def\csname PYG@tok@s1\endcsname{\def\PYG@tc##1{\textcolor[rgb]{0.25,0.44,0.63}{##1}}}
\expandafter\def\csname PYG@tok@ow\endcsname{\let\PYG@bf=\textbf\def\PYG@tc##1{\textcolor[rgb]{0.00,0.44,0.13}{##1}}}
\expandafter\def\csname PYG@tok@o\endcsname{\def\PYG@tc##1{\textcolor[rgb]{0.40,0.40,0.40}{##1}}}
\expandafter\def\csname PYG@tok@gp\endcsname{\let\PYG@bf=\textbf\def\PYG@tc##1{\textcolor[rgb]{0.78,0.36,0.04}{##1}}}
\expandafter\def\csname PYG@tok@sb\endcsname{\def\PYG@tc##1{\textcolor[rgb]{0.25,0.44,0.63}{##1}}}
\expandafter\def\csname PYG@tok@mb\endcsname{\def\PYG@tc##1{\textcolor[rgb]{0.13,0.50,0.31}{##1}}}
\expandafter\def\csname PYG@tok@nb\endcsname{\def\PYG@tc##1{\textcolor[rgb]{0.00,0.44,0.13}{##1}}}
\expandafter\def\csname PYG@tok@kd\endcsname{\let\PYG@bf=\textbf\def\PYG@tc##1{\textcolor[rgb]{0.00,0.44,0.13}{##1}}}

\def\PYGZbs{\char`\\}
\def\PYGZus{\char`\_}
\def\PYGZob{\char`\{}
\def\PYGZcb{\char`\}}
\def\PYGZca{\char`\^}
\def\PYGZam{\char`\&}
\def\PYGZlt{\char`\<}
\def\PYGZgt{\char`\>}
\def\PYGZsh{\char`\#}
\def\PYGZpc{\char`\%}
\def\PYGZdl{\char`\$}
\def\PYGZhy{\char`\-}
\def\PYGZsq{\char`\'}
\def\PYGZdq{\char`\"}
\def\PYGZti{\char`\~}
% for compatibility with earlier versions
\def\PYGZat{@}
\def\PYGZlb{[}
\def\PYGZrb{]}
\makeatother

\renewcommand\PYGZsq{\textquotesingle}

\begin{document}

\maketitle
\tableofcontents
\phantomsection\label{index::doc}


Contents:


\chapter{Install}
\label{install:welcome-to-eda5-project-s-documentation}\label{install:install}\label{install::doc}
This is where you write how to get a new laptop to run this project.


\chapter{Deploy}
\label{deploy::doc}\label{deploy:deploy}
This is where you describe how the project is deployed in production.


\chapter{Developing with Docker}
\label{docker_ec2::doc}\label{docker_ec2:developing-with-docker}
You can develop your application in a \href{https://www.docker.com/}{Docker} container for simpler deployment onto bare Linux machines later. This instruction assumes an \href{http://aws.amazon.com/}{Amazon Web Services} EC2 instance, but it should work on any machine with Docker \textgreater{} 1.3 and \href{https://docs.docker.com/compose/}{Docker compose} installed.


\section{Setting up}
\label{docker_ec2:docker-compose}\label{docker_ec2:setting-up}
Docker encourages running one container for each process. This might mean one container for your web server, one for Django application and a third for your database. Once you're happy composing containers in this way you can easily add more, such as a \href{http://redis.io/}{Redis} cache.

The Docker compose tool (previously known as \href{http://www.fig.sh/}{fig}) makes linking these containers easy. An example set up for your cookiecutter-django project might look like this:

\begin{Verbatim}[commandchars=\\\{\}]
webapp/ \PYGZsh{} Your cookiecutter project would be in here
    Dockerfile
    ...
database/
    Dockerfile
    ...
webserver/
    Dockerfile
    ...
docker\PYGZhy{}compose.yml
\end{Verbatim}

Each component of your application would get its own \href{https://docs.docker.com/reference/builder/}{Dockerfile}. The rest of this example assumes you are using the \href{https://registry.hub.docker.com/\_/postgres/}{base postgres image} for your database. Your database settings in \emph{config/common.py} might then look something like:

\begin{Verbatim}[commandchars=\\\{\}]
\PYG{n}{DATABASES} \PYG{o}{=} \PYG{p}{\PYGZob{}}
        \PYG{l+s}{\PYGZsq{}}\PYG{l+s}{default}\PYG{l+s}{\PYGZsq{}}\PYG{p}{:} \PYG{p}{\PYGZob{}}
            \PYG{l+s}{\PYGZsq{}}\PYG{l+s}{ENGINE}\PYG{l+s}{\PYGZsq{}}\PYG{p}{:} \PYG{l+s}{\PYGZsq{}}\PYG{l+s}{django.db.backends.postgresql\PYGZus{}psycopg2}\PYG{l+s}{\PYGZsq{}}\PYG{p}{,}
            \PYG{l+s}{\PYGZsq{}}\PYG{l+s}{NAME}\PYG{l+s}{\PYGZsq{}}\PYG{p}{:} \PYG{l+s}{\PYGZsq{}}\PYG{l+s}{postgres}\PYG{l+s}{\PYGZsq{}}\PYG{p}{,}
            \PYG{l+s}{\PYGZsq{}}\PYG{l+s}{USER}\PYG{l+s}{\PYGZsq{}}\PYG{p}{:} \PYG{l+s}{\PYGZsq{}}\PYG{l+s}{postgres}\PYG{l+s}{\PYGZsq{}}\PYG{p}{,}
            \PYG{l+s}{\PYGZsq{}}\PYG{l+s}{HOST}\PYG{l+s}{\PYGZsq{}}\PYG{p}{:} \PYG{l+s}{\PYGZsq{}}\PYG{l+s}{database}\PYG{l+s}{\PYGZsq{}}\PYG{p}{,}
            \PYG{l+s}{\PYGZsq{}}\PYG{l+s}{PORT}\PYG{l+s}{\PYGZsq{}}\PYG{p}{:} \PYG{l+m+mi}{5432}\PYG{p}{,}
        \PYG{p}{\PYGZcb{}}
    \PYG{p}{\PYGZcb{}}
\end{Verbatim}

The \href{https://docs.docker.com/compose/\#compose-documentation}{Docker compose documentation} explains in detail what you can accomplish in the \emph{docker-compose.yml} file, but an example configuration might look like this:

\begin{Verbatim}[commandchars=\\\{\}]
\PYG{l+lScalar+lScalarPlain}{database}\PYG{p+pIndicator}{:}
    \PYG{l+lScalar+lScalarPlain}{build}\PYG{p+pIndicator}{:} \PYG{l+lScalar+lScalarPlain}{database}
\PYG{l+lScalar+lScalarPlain}{webapp}\PYG{p+pIndicator}{:}
    \PYG{l+lScalar+lScalarPlain}{build}\PYG{p+pIndicator}{:} \PYG{l+lScalar+lScalarPlain}{webapp}\PYG{p+pIndicator}{:}
    \PYG{l+lScalar+lScalarPlain}{command}\PYG{p+pIndicator}{:} \PYG{l+lScalar+lScalarPlain}{/usr/bin/python3.4}\PYG{l+lScalar+lScalarPlain}{ }\PYG{l+lScalar+lScalarPlain}{manage.py}\PYG{l+lScalar+lScalarPlain}{ }\PYG{l+lScalar+lScalarPlain}{runserver}\PYG{l+lScalar+lScalarPlain}{ }\PYG{l+lScalar+lScalarPlain}{0.0.0.0:8000} \PYG{c+c1}{\PYGZsh{} dev setting}
    \PYG{c+c1}{\PYGZsh{} command: gunicorn \PYGZhy{}b 0.0.0.0:8000 wsgi:application \PYGZsh{} production setting}
    \PYG{l+lScalar+lScalarPlain}{volumes}\PYG{p+pIndicator}{:}
        \PYG{p+pIndicator}{\PYGZhy{}} \PYG{l+lScalar+lScalarPlain}{webapp/your\PYGZus{}project\PYGZus{}name:/path/to/container/workdir/}
    \PYG{l+lScalar+lScalarPlain}{links}\PYG{p+pIndicator}{:}
        \PYG{p+pIndicator}{\PYGZhy{}} \PYG{l+lScalar+lScalarPlain}{database}
\PYG{l+lScalar+lScalarPlain}{webserver}\PYG{p+pIndicator}{:}
    \PYG{l+lScalar+lScalarPlain}{build}\PYG{p+pIndicator}{:} \PYG{l+lScalar+lScalarPlain}{webserver}
    \PYG{l+lScalar+lScalarPlain}{ports}\PYG{p+pIndicator}{:}
        \PYG{p+pIndicator}{\PYGZhy{}} \PYG{l+s}{\PYGZdq{}}\PYG{l+s}{80:80}\PYG{l+s}{\PYGZdq{}}
        \PYG{p+pIndicator}{\PYGZhy{}} \PYG{l+s}{\PYGZdq{}}\PYG{l+s}{443:443}\PYG{l+s}{\PYGZdq{}}
    \PYG{l+lScalar+lScalarPlain}{links}\PYG{p+pIndicator}{:}
        \PYG{p+pIndicator}{\PYGZhy{}} \PYG{l+lScalar+lScalarPlain}{webapp}
\end{Verbatim}

We'll ignore the webserver for now (you'll want to comment that part out while we do). A working Dockerfile to run your cookiecutter application might look like this:

\begin{Verbatim}[commandchars=\\\{\}]
FROM ubuntu:14.04
ENV REFRESHED\PYGZus{}AT 2015\PYGZhy{}01\PYGZhy{}13

\PYGZsh{} update packages and prepare to build software
RUN [\PYGZdq{}apt\PYGZhy{}get\PYGZdq{}, \PYGZdq{}update\PYGZdq{}]
RUN [\PYGZdq{}apt\PYGZhy{}get\PYGZdq{}, \PYGZdq{}\PYGZhy{}y\PYGZdq{}, \PYGZdq{}install\PYGZdq{}, \PYGZdq{}build\PYGZhy{}essential\PYGZdq{}, \PYGZdq{}vim\PYGZdq{}, \PYGZdq{}git\PYGZdq{}, \PYGZdq{}curl\PYGZdq{}]
RUN [\PYGZdq{}locale\PYGZhy{}gen\PYGZdq{}, \PYGZdq{}en\PYGZus{}GB.UTF\PYGZhy{}8\PYGZdq{}]

\PYGZsh{} install latest python
RUN [\PYGZdq{}apt\PYGZhy{}get\PYGZdq{}, \PYGZdq{}\PYGZhy{}y\PYGZdq{}, \PYGZdq{}build\PYGZhy{}dep\PYGZdq{}, \PYGZdq{}python3\PYGZhy{}dev\PYGZdq{}, \PYGZdq{}python3\PYGZhy{}imaging\PYGZdq{}]
RUN [\PYGZdq{}apt\PYGZhy{}get\PYGZdq{}, \PYGZdq{}\PYGZhy{}y\PYGZdq{}, \PYGZdq{}install\PYGZdq{}, \PYGZdq{}python3\PYGZhy{}dev\PYGZdq{}, \PYGZdq{}python3\PYGZhy{}imaging\PYGZdq{}, \PYGZdq{}python3\PYGZhy{}pip\PYGZdq{}]

\PYGZsh{} prepare postgreSQL support
RUN [\PYGZdq{}apt\PYGZhy{}get\PYGZdq{}, \PYGZdq{}\PYGZhy{}y\PYGZdq{}, \PYGZdq{}build\PYGZhy{}dep\PYGZdq{}, \PYGZdq{}python3\PYGZhy{}psycopg2\PYGZdq{}]

\PYGZsh{} move into our working directory
\PYGZsh{} ADD must be after chown see http://stackoverflow.com/a/26145444/1281947
RUN [\PYGZdq{}groupadd\PYGZdq{}, \PYGZdq{}python\PYGZdq{}]
RUN [\PYGZdq{}useradd\PYGZdq{}, \PYGZdq{}python\PYGZdq{}, \PYGZdq{}\PYGZhy{}s\PYGZdq{}, \PYGZdq{}/bin/bash\PYGZdq{}, \PYGZdq{}\PYGZhy{}m\PYGZdq{}, \PYGZdq{}\PYGZhy{}g\PYGZdq{}, \PYGZdq{}python\PYGZdq{}, \PYGZdq{}\PYGZhy{}G\PYGZdq{}, \PYGZdq{}python\PYGZdq{}]
ENV HOME /home/python
WORKDIR /home/python
RUN [\PYGZdq{}chown\PYGZdq{}, \PYGZdq{}\PYGZhy{}R\PYGZdq{}, \PYGZdq{}python:python\PYGZdq{}, \PYGZdq{}/home/python\PYGZdq{}]
ADD ./ /home/python

\PYGZsh{} manage requirements
ENV REQUIREMENTS\PYGZus{}REFRESHED\PYGZus{}AT 2015\PYGZhy{}02\PYGZhy{}25
RUN [\PYGZdq{}pip3\PYGZdq{}, \PYGZdq{}install\PYGZdq{}, \PYGZdq{}\PYGZhy{}r\PYGZdq{}, \PYGZdq{}requirements.txt\PYGZdq{}]

\PYGZsh{} uncomment the line below to use container as a non\PYGZhy{}root user
USER python:python
\end{Verbatim}

Running \emph{sudo docker-compose build} will follow the instructions in your \emph{docker-compose.yml} file and build the database container, then your webapp, before mounting your cookiecutter project files as a volume in the webapp container and linking to the database. Our example yaml file runs in development mode but changing it to production mode is as simple as commenting out the line using \emph{runserver} and uncommenting the line using \emph{gunicorn}.

Both are set to run on port \emph{0.0.0.0:8000}, which is where the Docker daemon will discover it. You can now run \emph{sudo docker-compose up} and browse to \emph{localhost:8000} to see your application running.


\section{Deployment}
\label{docker_ec2:deployment}
You'll need a webserver container for deployment. An example setup for \href{http://wiki.nginx.org/Main}{Nginx} might look like this:

\begin{Verbatim}[commandchars=\\\{\}]
FROM ubuntu:14.04
ENV REFRESHED\PYGZus{}AT 2015\PYGZhy{}02\PYGZhy{}11

\PYGZsh{} get the nginx package and set it up
RUN [\PYGZdq{}apt\PYGZhy{}get\PYGZdq{}, \PYGZdq{}update\PYGZdq{}]
RUN [\PYGZdq{}apt\PYGZhy{}get\PYGZdq{}, \PYGZdq{}\PYGZhy{}y\PYGZdq{}, \PYGZdq{}install\PYGZdq{}, \PYGZdq{}nginx\PYGZdq{}]

\PYGZsh{} forward request and error logs to docker log collector
RUN ln \PYGZhy{}sf /dev/stdout /var/log/nginx/access.log
RUN ln \PYGZhy{}sf /dev/stderr /var/log/nginx/error.log
VOLUME [\PYGZdq{}/var/cache/nginx\PYGZdq{}]
EXPOSE 80 443

\PYGZsh{} load nginx conf
ADD ./site.conf /etc/nginx/sites\PYGZhy{}available/your\PYGZus{}cookiecutter\PYGZus{}project
RUN [\PYGZdq{}ln\PYGZdq{}, \PYGZdq{}\PYGZhy{}s\PYGZdq{}, \PYGZdq{}/etc/nginx/sites\PYGZhy{}available/your\PYGZus{}cookiecutter\PYGZus{}project\PYGZdq{}, \PYGZdq{}/etc/nginx/sites\PYGZhy{}enabled/your\PYGZus{}cookiecutter\PYGZus{}project\PYGZdq{}]
RUN [\PYGZdq{}rm\PYGZdq{}, \PYGZdq{}\PYGZhy{}rf\PYGZdq{}, \PYGZdq{}/etc/nginx/sites\PYGZhy{}available/default\PYGZdq{}]

\PYGZsh{}start the server
CMD [\PYGZdq{}nginx\PYGZdq{}, \PYGZdq{}\PYGZhy{}g\PYGZdq{}, \PYGZdq{}daemon off;\PYGZdq{}]
\end{Verbatim}

That Dockerfile assumes you have an Nginx conf file named \emph{site.conf} in the same directory as the webserver Dockerfile. A very basic example, which forwards traffic onto the development server or gunicorn for processing, would look like this:

\begin{Verbatim}[commandchars=\\\{\}]
\PYGZsh{} see http://serverfault.com/questions/577370/how\PYGZhy{}can\PYGZhy{}i\PYGZhy{}use\PYGZhy{}environment\PYGZhy{}variables\PYGZhy{}in\PYGZhy{}nginx\PYGZhy{}conf\PYGZsh{}comment730384\PYGZus{}577370
upstream localhost \PYGZob{}
    server webapp\PYGZus{}1:8000;
\PYGZcb{}
server \PYGZob{}
    location / \PYGZob{}
        proxy\PYGZus{}pass http://localhost;
    \PYGZcb{}
\PYGZcb{}
\end{Verbatim}

Running \emph{sudo docker-compose build webserver} will build your server container. Running \emph{sudo docker-compose up} will now expose your application directly on \emph{localhost} (no need to specify the port number).


\section{Building and running your app on EC2}
\label{docker_ec2:building-and-running-your-app-on-ec2}
All you now need to do to run your app in production is:
\begin{itemize}
\item {} 
Create an empty EC2 Linux instance (any Linux machine should do).

\item {} 
Install your preferred source control solution, Docker and Docker compose on the news instance.

\item {} 
Pull in your code from source control. The root directory should be the one with your \emph{docker-compose.yml} file in it.

\item {} 
Run \emph{sudo docker-compose build} and \emph{sudo docker-compose up}.

\item {} 
Assign an \href{https://aws.amazon.com/articles/1346}{Elastic IP address} to your new machine.

\end{itemize}
\begin{itemize}
\item {} 
Point your domain name to the elastic IP.

\end{itemize}

\textbf{Be careful with Elastic IPs} because, on the AWS free tier, if you assign one and then stop the machine you will incur charges while the machine is down (presumably because you're preventing them allocating the IP to someone else).


\section{Security advisory}
\label{docker_ec2:security-advisory}
The setup described in this instruction will get you up-and-running but it hasn't been audited for security. If you are running your own setup like this it is always advisable to, at a minimum, examine your application with a tool like \href{https://www.owasp.org/index.php/OWASP\_Zed\_Attack\_Proxy\_Project}{OWASP ZAP} to see what security holes you might be leaving open.


\chapter{Struktura projekt}
\label{struktura::doc}\label{struktura:owasp-zap}\label{struktura:struktura-projekt}

\section{Vidiki uporabe}
\label{struktura:vidiki-uporabe}

\subsection{Pogled za Izvajalca Del}
\label{view_za_izvajalca::doc}\label{view_za_izvajalca:pogled-za-izvajalca-del}

\subsubsection{IZVEDBA DEL : DELOVNI-NALOGI}
\label{view_za_izvajalca:izvedba-del-delovni-nalogi}\begin{description}
\item[{Statusi delovnega-naloga:}] \leavevmode\begin{itemize}
\item {} 
V ČAKANJU

\item {} 
V PLANU

\item {} 
V REŠEVANJU

\item {} 
ZAKLJUČENI

\end{itemize}

\end{description}


\paragraph{V ČAKANJU}
\label{view_za_izvajalca:v-cakanju}\begin{quote}
\begin{description}
\item[{Namen:}] \leavevmode\begin{itemize}
\item {} 
Čakanje na potrditev nosilca izvedbe-del po delovnem-nalogu.

\end{itemize}

\item[{Polja:}] \leavevmode\begin{itemize}
\item {} 
delovni nalog (oznaka)

\item {} 
opravilo

\item {} 
rok izvedbe

\item {} 
nosilec

\item {} 
nadzor

\end{itemize}

\item[{Posebnosti prikaza:}] \leavevmode\begin{itemize}
\item {} 
delovne-naloge vidi samo nosilec (opcijsko vključi v nastavitev)

\end{itemize}

\item[{Aktivnosti:}] \leavevmode\begin{itemize}
\item {} 
kliknemo na delovni-nalog

\item {} \begin{description}
\item[{vnesemo ``planirani datum izvedbe del'' po delovnem-nalogu}] \leavevmode\begin{itemize}
\item {} 
datum ne sme biti čez rok-izvedbe-opravila

\item {} 
razpon dni med datumom ``v planu'' in ``rok izvedbe'' ne sme biti večji od ``št\_dni'' \textless{}-- definiraj v opravilih

\end{itemize}

\end{description}

\item {} 
možnost vpisa zaznamkov (posebnosti) za nadzor

\item {} 
potrditev s klikom na ``POTRDI''

\item {} 
po uspešni potrditvi (validacija) se status delovnega-naloga spremeni = ``V PLANU''

\end{itemize}

\end{description}
\end{quote}


\paragraph{V PLANU}
\label{view_za_izvajalca:v-planu}\begin{description}
\item[{Aktivnosti:}] \leavevmode\begin{itemize}
\item {} 
vnos prvega dela = sprememba statusa delovnega-naloga = ``V REŠEVANJU''

\end{itemize}

\end{description}


\paragraph{V REŠEVANJU}
\label{view_za_izvajalca:v-resevanju}\begin{description}
\item[{Aktivnosti:}] \leavevmode\begin{itemize}
\item {} 
s pritiksom na zaključi delovni-nalog se status spremeni = ``ZAKLJUČENO''

\item {} 
delovni-nalog se avtomatsko natisne. Kasnejše tiskanje je možno pod

\end{itemize}

zaključeni delovni-nalogi

\end{description}


\paragraph{ZAKLJUČENI}
\label{view_za_izvajalca:zakljuceni}\begin{description}
\item[{Podatki:}] \leavevmode\begin{itemize}
\item {} 
Podatki o OPRAVILU in DELOVNEM NALOGU

\item {} 
\item {} 
\end{itemize}

\item[{Dodatne opcije:}] \leavevmode\begin{itemize}
\item {} 
možnost tiskanja delovnega naloga

\end{itemize}

\end{description}


\section{Aplikacije}
\label{struktura:aplikacije}

\subsection{Delovni Nalogi}
\label{delovninalogi::doc}\label{delovninalogi:delovni-nalogi}

\subsubsection{Namen/Opis}
\label{delovninalogi:namen-opis}\begin{itemize}
\item {} 
Elektronski Delovni Nalog

\item {} 
Dnevniki in Obračuni

\item {} 
Servisna Knjiga

\item {} 
Arhiv Prejete Dokumentacije

\end{itemize}


\subsubsection{models}
\label{delovninalogi:models}
\begin{Verbatim}[commandchars=\\\{\}]
\PYG{n}{asdasd}
\PYG{n}{asd}
\end{Verbatim}

asd


\chapter{Indices and tables}
\label{index:indices-and-tables}\begin{itemize}
\item {} 
\DUspan{xref,std,std-ref}{genindex}

\item {} 
\DUspan{xref,std,std-ref}{modindex}

\item {} 
\DUspan{xref,std,std-ref}{search}

\end{itemize}



\renewcommand{\indexname}{Index}
\printindex
\end{document}
